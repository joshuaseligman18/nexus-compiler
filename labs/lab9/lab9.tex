\documentclass[letterpaper, 10pt, DIV=13]{scrartcl}
\usepackage[T1]{fontenc}
\usepackage[english]{babel}
\usepackage{amsmath, amsfonts, amsthm, xfrac}
\usepackage{listings}
\usepackage{color}
\usepackage{longtable}
\usepackage{qtree}

\numberwithin{equation}{section}
\numberwithin{figure}{section}
\numberwithin{table}{section}

\usepackage{sectsty}
\allsectionsfont{\normalfont\scshape} % Make all section titles in default font and small caps.

\usepackage{fancyhdr} % Custom headers and footers
\pagestyle{fancyplain} % Makes all pages in the document conform to the custom headers and footers

\fancyhead{} % No page header - if you want one, create it in the same way as the footers below
\fancyfoot[L]{} % Empty left footer
\fancyfoot[C]{} % Empty center footer
\fancyfoot[R]{\thepage} % Page numbering for right footer

\renewcommand{\headrulewidth}{0pt} % Remove header underlines
\renewcommand{\footrulewidth}{0pt} % Remove footer underlines
\setlength{\headheight}{13.6pt} % Customize the height of the header

\setlength\parindent{0pt}
\pagenumbering{gobble}

\title {
	\normalfont
	\huge{Lab 9} \\
	\vspace{10pt}
	\large{CMPT 432 - Spring 2023 | Dr. Labouseur}
}

\author{\normalfont Josh Seligman | joshua.seligman1@marist.edu}

\pagenumbering{arabic}

\definecolor{mygreen}{rgb}{0,0.6,0}
\definecolor{mygray}{rgb}{0.5,0.5,0.5}
\definecolor{mymauve}{rgb}{0.58,0,0.82}
\lstset{
  backgroundcolor=\color{white},   % choose the background color
  basicstyle=\footnotesize,        % size of fonts used for the code
  breaklines=true,                 % automatic line breaking only at whitespace
  captionpos=b,                    % sets the caption-position to bottom
  commentstyle=\color{mygreen},    % comment style
  escapeinside={\%*}{*},          % if you want to add LaTeX within your code
  keywordstyle=\color{blue},       % keyword style
  stringstyle=\color{mymauve},     % string literal style
}

\begin{document}
\maketitle

\section{Crafting a Compiler}
\subsection{Exercise 5.5}
Transform the following grammar into LL(1) form using the techniques
presented in Section 5.5:
\begin{enumerate}
    \item DeclList $\rightarrow$ DeclList; Decl
    \item $\mid$ Decl
    \item Decl $\rightarrow$ IdList : Type
    \item IdList $\rightarrow$ IdList, id
    \item $\mid$ id
    \item Type $\rightarrow$ ScalarType
    \item $\mid$ array ( ScalarTypeList ) of Type
    \item ScalarType $\rightarrow$ id
    \item $\mid$ Bound . . Bound
    \item Bound $\rightarrow$ Sign intconstant
    \item $\mid$ id
    \item Sign $\rightarrow$ +
    \item $\mid$ - 
    \item $\mid$ $\lambda$ 
    \item ScalarTypelist $\rightarrow$ ScalarTypeList, ScalarType
    \item $\mid$ ScalarType
\end{enumerate}

Removal of left recursion for DeclList, IdList, ScalarTypeList and also
modified the base cases to include $\epsilon$ to make them LL(1) because
otherwise multiple productions match their first token Create the productions
for BoundId and BoundIdFollow because ScalarType and Bound both match id and
are not LL(1), so combine them to become LL(1).
\begin{enumerate}
    \item DeclList $\rightarrow$ Decl DeclNext
    \item DeclNext $\rightarrow$ ; DeclList
    \item $\mid$ $\epsilon$
    \item Decl $\rightarrow$ IdList : Type
    \item IdList $\rightarrow$ id IdNext
    \item IdNext $\rightarrow$ , IdList
    \item $\mid$ $\epsilon$
    \item Type $\rightarrow$ ScalarType
    \item $\mid$ array ( ScalarTypeList ) of Type
    \item BoundId $\rightarrow$ id BoundIdFollow
    \item BoundIdFollow $\rightarrow$ . . id
    \item $\mid$ $\epsilon$
    \item ScalarType $\rightarrow$ BoundId
    \item $\mid$ Sign intconstant
    \item Sign $\rightarrow$ +
    \item $\mid$ - 
    \item $\mid$ $\lambda$ 
    \item ScalarTypelist $\rightarrow$ ScalarType, ScalarTypeList
    \item ScalarTypeNext $\rightarrow$ , ScalarTypeList
    \item $\mid$ $\epsilon$
\end{enumerate}

\section{Dragon}
\subsection{Exercise 4.2.1}
\end{document}
